\documentclass[a4paper]{article}
\usepackage{fullpage}
\usepackage{amsmath}
\usepackage{amssymb}
\usepackage{textcomp}
\usepackage[utf8]{inputenc}
\usepackage[T1]{fontenc}
\textheight=10in
\raggedright

\usepackage{fancyhdr}
\pagestyle{fancy}
\fancyhf{}
\renewcommand{\headrulewidth}{0pt}
\cfoot{Jason Swindle \hfill \thepage}

\def\bull{\vrule height 0.8ex width .7ex depth -.1ex }

% DEFINITIONS FOR RESUME %%%%%%%%%%%%%%%%%%%%%%%

\newcommand{\area} [2] {
    \vspace*{-9pt}
    \begin{verse}
        \textbf{#1}   #2
    \end{verse}
}

\newcommand{\lineunder} {
    \vspace*{-8pt} \\
    \hspace*{-18pt} \hrulefill \\
}

\newcommand{\header} [1] {
    {\hspace*{-18pt}\vspace*{6pt} \textsc{#1}}
    \vspace*{-6pt} \lineunder
}

\newcommand{\employer} [3] {
    { \textbf{#1} (#2)\\ \underline{\textbf{\emph{#3}}}\\  }
}

\newcommand{\contact} [3] {
    \vspace*{-10pt}
    \begin{center}
        {\Huge \scshape {#1}}\\
        #2 \\ #3
    \end{center}
    \vspace*{-8pt}
}

\newenvironment{achievements}{
    \begin{list}
        {$\bullet$}{\topsep 0pt \itemsep -2pt}}{\vspace*{4pt}
    \end{list}
}

\newcommand{\schoolwithcourses} [4] {
    \textbf{#1} #2 $\bullet$ #3\\
    #4 \\
    \vspace*{5pt}
}

\newcommand{\school} [4] {
    \textbf{#1} #2 $\bullet$ #3\\
    #4 \\
}
% END RESUME DEFINITIONS %%%%%%%%%%%%%%%%%%%%%%%

\begin{document}
\vspace*{-40pt}

%==== Profile ====%
\vspace*{-10pt}
\begin{center}
    {\Huge \scshape {Jason Swindle}}\\[1\baselineskip]
    jason.swindle@gmail.com $\cdot$ (256) 275 - 4228 $\cdot$ https://resume.swindle.me\\[1\baselineskip]
\end{center}

%==== Experience ====%
\header{Experience}
\vspace{1mm}

\textbf{Amazon Web Services, Inc} \hfill Seattle, Washington\\
\textit{Specialist Technical Account Manager for Container Technologies} \hfill 2018-09-17 - Present\\
\vspace{-1mm}
\begin{itemize} \itemsep 1pt
    \item First Specialist Technical Account Manager (STAM) for Containers in Amazon / AWS, and helped shaped the role and direction of the STAM organization.
    \item Provided guidance for highly scalable, flexible and resilient cloud workloads for enterprise AWS customers using a broad range of technologies and skills
    \item Technical liaison between customers, service teams and support. This included being the voice of the customer during product creation, escalations, and product improvements.
    \item Provided a proof-of-concept monitoring infrastructure which monitored 1.5 million vCPUs and 55,000+ Amazon EC2 servers with a 1-minute monitoring service level agreement (SLA). This setup included the use of: Prometheus for the collection and storage of metrics, Grafana for the custom dashboards, Docker for the deployment, Amazon EC2 for the servers, Amazon Application Load Balancer for load balancing request to Grafana, and Prometheus Node Exporter for server level metrics.
    \item Worked closely with a Fortune 500 customer\textsc{\char13}s engineering team to triage and unblock them from a production outage, and identified a software bug that could impact other customers.
    \item Joint contributor to the \textquotedbl{}EKS Log Collector\textquotedbl{}, which is now included in the Amazon Elastic Kubernetes Service AMI and Github repository. This helped decrease case resolve time and reduced the total number of correspondences.
\end{itemize}
\textbf{Amazon Web Services, Inc} \hfill Seattle, Washington\\
\textit{Cloud Support Engineer II} \hfill 2016-10-01 - 2018-09-17\\
\vspace{-1mm}
\begin{itemize} \itemsep 1pt
    \item Served as the highest tier of support to mitigate escalations to the respective service team, and improve turn around time on customer issues.
    \item Presented at the ECS Global Subject Matter Expert Summit that had solution architects, product leadership, and premium support.
    \item Point of Contact for ECS / ECR / Fargate / EKS / Docker in support for the service teams, solution architects, and leadership.
    \item The preferred engineer and single point of contact for all things ECS / ECR / Docker for a large customer that successfully launched a feature used by up to 100 million users daily.
    \item Traveled to Berlin to give a presentation on ECS and Prometheus monitoring to a fast growth start-up.
    \item Helped improve process in the support organization by strengthening the relationship with the documentation team to lower customer contacts and clarify complex topics, and led project to create and update internal tooling for premium support
    \item Helped launch Amazon Fargate, and Amazon Elastic Kubernetes Service: created and delivered training material for premium support on new services, collaborated with the service team to advocate for customer features and served as liaison between service team and premium support
    \item All job functions of cloud support engineer I continued to be served.
\end{itemize}
\textbf{Amazon Web Services, Inc} \hfill Seattle, Washington\\
\textit{Cloud Support Engineer I} \hfill 2015-04-13 - 2016-10-01\\
\vspace{-1mm}
\begin{itemize} \itemsep 1pt
    \item Was the only AWS premium support Engineer to represent AWS at DockerCon 2016 (Seattle) in 1-on-1 interactions with customers to understand their needs / gather feedback on ECS or AWS.
    \item Provided best practices and conducted troubleshooting for high severity issues for AWS customers who use AWS deployment management services such as (but not limited to): Elastic Beanstalk, EC2 Container Service (ECS), EC2 Container Registry (ECR), CloudFormation, OpsWorks, CodeDeploy, CodePipeline, and CodeCommit.
    \item One of three premium support Engineers who met bi-weekly with ECS / ECR product leadership to act as a customer\textsc{\char13}s advocate which included highlighting: feature requests, bug reports, customer complaints, and overall inter-team communication improvements.
    \item Collaborated with AWS premium support of Sydney office to create introduction, intermediate, advanced, and Subject Matter Expert (SME) level trainings for ECS, ECR, and Docker.
    \item Inaugural accredited ECS subject matter experts (SME) at Amazon. This accreditation was validated and approved by the creators of AWS ECS. 
    \item Participated at \textquotedbl{}Ask an Architect\textquotedbl{} at AWS Pop-up Loft in San Francisco, providing 1-on-1, 60 minute sessions to attendees who needed technical deep dives into AWS and Docker, or attendees who expressed interest in learning the fundamentals of cloud computing and DevOps.
    \item Attended the ECS Global Subject Matter Expert Summit and represented the needs of the customers and support to solution architects and product leadership.
\end{itemize}

%==== Skills ====%
\header{Skills}
\textbf{Deployment Methodologies}\\
\vspace{-1mm}
\begin{itemize} \itemsep 1pt
    \item Containers / Docker / Kubernetes / Amazon ECS
    \item Continuous Integration \& Continuous Delivery
    \item SaltStack / Ansible / Chef / Puppet
    \item GitOps / WeaveWorks Flux
    \item Terraform / Packer
\end{itemize}

\textbf{AWS Products Specialty}\\
\vspace{-1mm}
\begin{itemize} \itemsep 1pt
    \item Amazon Elastic Kubernetes Service
    \item Amazon Elastic Container Service \& AWS Fargate
    \item Amazon Elastic Container Registry
    \item AWS Cloud Formation
\end{itemize}

%==== Projects ====%
\header{Projects}
\textbf{AWS ECS Local DNS Cache} \hfill github.com/JasonSwindle/ecs-local-dns-cache\\
\textit{Created to allow faster DNS resolution for cached items and to mitigate containerized workloads from overloaded upstream DNS providers.} \\[1\baselineskip]

% \vspace{-1mm}
% \begin{itemize} \itemsep 1pt
%	\item CoreDNS; Used for DNS caching.
%	\item Docker; Used for the application encapsulation.
%	\item AWS ECS; Used for the container orchestration of the Docker Containers.
%    \item Bash; Used as the primary scripting language.
%\end{itemize}

\textbf{EKS Log Collector} \hfill github.com/awslabs/amazon-eks-ami\\
\textit{Created to expedite the handling of a customer case and reduce the number of case correspondence. The script collects a broad range of data points for the end-user automatically. Previously this process  was manual and error-prone.} \\[1\baselineskip]

%\vspace{-1mm}
%\begin{itemize} \itemsep 1pt
%	\item CoreDNS; Used for DNS caching.
%	\item Docker; Used for the application encapsulation.
%	\item AWS ECS; Used for the container orchestration of the Docker Containers.
%    \item Bash; Used as the primary scripting language.
%\end{itemize}

%==== Education ====%
\header{Education}
\textbf{University of North Alabama}\hfill Florence, Alabama\\
Bachelor Computer Information Systems \hfill 2004-06-01 - 2008-12-01\\
\vspace{2mm}

%==== Awards ====%
\header{Awards}
\textbf{SaltStack Certified Engineer (SSCE)} \hfill SaltStack, Inc.\\
13th SaltStack Certified Engineer Globally (0x2E6A0C88) \hfill 2014-01-29\\
\vspace*{2mm}

\end{document}
