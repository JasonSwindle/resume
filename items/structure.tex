% The `geometry` package is used to set the margins and page layout.
\usepackage[
    paper=a4paper, % Sets the paper size to A4.
    margin=50pt, % Sets the margin to 50pt on all sides.
    nomarginpar, % Disables margin paragraphs.
    nohead % Disables the header.
]{geometry}

% The following packages are used for various formatting purposes.
\usepackage{
    tabularx, % Advanced table formatting.
    multirow, % For table cells that span multiple rows.
    enumitem, % For customizing lists.
    titlesec, % For customizing section titles.
    xifthen, % For conditional commands.
    hyperref % For creating hyperlinks in the document.
}

\setlist[itemize]{
    topsep=5pt,
    parsep=0pt,
    itemsep=6pt
}

% PDF Stuff
\usepackage{pslatex}
\pdfmajorversion=2
\pdfminorversion=0
\pdfobjcompresslevel=2
\pdfinclusionerrorlevel=1

\hypersetup{
    pdfauthor={\yourname},%
    pdftitle={\yourname - Resume},
    pdfsubject={Resume},
    pdfkeywords={Resume, AWS, GCP, Google, DevOps, TAM},
    pdfproducer={LaTeX},
    pdfcreator={pdfLaTeX},
    % pdfcreationdate={\today},
    allcolors=black,
    hidelinks,
    bookmarks=true
}

\usepackage[
    noeditdata=true,
    noproducerdata=false]{pdfprivacy}

% Roman numerals
\newcommand{\RNum} [1] {
    \uppercase\expandafter{\romannumeral #1\relax}
}

% Ordinal numbers
\usepackage[super]{nth}

% Phone number formatting
\usepackage{phonenumbers}

%\usepackage{fullpage}
\usepackage[utf8]{inputenc}
\usepackage[T1]{fontenc}
\usepackage[english]{babel}
\usepackage[english,printdayoff]{isodate}
\usepackage{fancyhdr}
\usepackage{multicol}

\raggedright

% Header Stuff
\pagestyle{fancy}
\fancyhf{}

% Fooder Stuff
\fancyfoot[L]{\yourname}
\fancyfoot[R]{\thepage}
\renewcommand{\headrulewidth}{0pt}

% Section formatting
\titleformat{\section}{\scshape\raggedright}{}{-15pt}{}
[\vspace{-9pt}\hspace*{-15pt}\titlerule]
\titlespacing{\section}{0pt}{0pt}{1pt}

% The `\roleheader` command is used to create a header for a role in the experience section.
% It takes five arguments:
%   #1: Company Name
%   #2: Location
%   #3: Title
%   #4: Start Date in 'YYYY-MM-DD' format
%   #5: End Date in 'YYYY-MM-DD' format. If left blank, it will print 'Present'.
\newcommand{\roleheader}[5]{
    \setlength\tabcolsep{0pt}
    \begin{tabularx}{\textwidth}{X r}
        \textbf{#1} & #2 \\
        #3 & \ifthenelse{\isempty{#5}}{\printdate{#4} to Present}{\daterange{#4}{#5}}
    \end{tabularx}
}

% The `\projectitem` command is used to create a project item in the projects section.
% It takes three arguments:
%   #1: Project Name
%   #2: URL
%   #3: Description
\newcommand{\projectitem}[3]{
    \setlength{\tabcolsep}{0pt}
    \begin{tabularx}{\textwidth}{ X r }
        \textbf{#1} & #2 \\
        \multicolumn{2}{ p{\textwidth} }{#3}
    \end{tabularx} \\[5pt]
}

% The `\educationitem` command is used to create an education item in the education section.
% It takes five arguments:
%   #1: University Name
%   #2: Location
%   #3: Degree
%   #4: Start Date in 'YYYY-MM-DD' format
%   #5: End Date in 'YYYY-MM-DD' format. If left blank, it will print 'Present'.
\newcommand{\educationitem}[5]{
    \setlength\tabcolsep{0pt}
    \begin{tabularx}{\textwidth}{X r}
        \textbf{#1} & #2 \\
        #3 & \ifthenelse{\isempty{#5}}{\printdate{#4} to Present}{\daterange{#4}{#5}}
    \end{tabularx}
}

% The `\profile` command is used to create the profile section at the top of the resume.
% It takes five arguments:
%   #1: Name
%   #2: Email Address
%   #3: Two-digit Country Code
%   #4: Phone Number
%   #5: Website. If left blank, it will be omitted.
\newcommand{\profile}[5]{
    \centering
    \begin{tabular}{ccc}
        \multicolumn{3}{c}{\Huge \scshape {#1}} \\ \\
        {\href{mailto:#2}{#2}} & {\phonenumber[country={#3},foreign]{#4}} & \href{#5}{#5} \\[5pt]
    \end{tabular}
}

% The `\awarditem` command is used to create an award item in the awards section.
% It takes four arguments:
%   #1: Award Name
%   #2: Award Provider
%   #3: Description
%   #4: Date in 'YYYY-MM-DD' format
\newcommand{\awarditem}[4]{
    \setlength\tabcolsep{0pt}
    \begin{tabularx}{\textwidth}{X r}
        \textbf{#1} & #2 \\
        #3 & \printdate{#4}
    \end{tabularx} \\[5pt]
}
